% Options for packages loaded elsewhere
\PassOptionsToPackage{unicode}{hyperref}
\PassOptionsToPackage{hyphens}{url}
\PassOptionsToPackage{dvipsnames,svgnames,x11names}{xcolor}
%
\documentclass[
  11pt,
]{article}
\usepackage{amsmath,amssymb}
\usepackage{setspace}
\usepackage{iftex}
\ifPDFTeX
  \usepackage[T1]{fontenc}
  \usepackage[utf8]{inputenc}
  \usepackage{textcomp} % provide euro and other symbols
\else % if luatex or xetex
  \usepackage{unicode-math} % this also loads fontspec
  \defaultfontfeatures{Scale=MatchLowercase}
  \defaultfontfeatures[\rmfamily]{Ligatures=TeX,Scale=1}
\fi
\usepackage{lmodern}
\ifPDFTeX\else
  % xetex/luatex font selection
\fi
% Use upquote if available, for straight quotes in verbatim environments
\IfFileExists{upquote.sty}{\usepackage{upquote}}{}
\IfFileExists{microtype.sty}{% use microtype if available
  \usepackage[]{microtype}
  \UseMicrotypeSet[protrusion]{basicmath} % disable protrusion for tt fonts
}{}
\makeatletter
\@ifundefined{KOMAClassName}{% if non-KOMA class
  \IfFileExists{parskip.sty}{%
    \usepackage{parskip}
  }{% else
    \setlength{\parindent}{0pt}
    \setlength{\parskip}{6pt plus 2pt minus 1pt}}
}{% if KOMA class
  \KOMAoptions{parskip=half}}
\makeatother
\usepackage{xcolor}
\usepackage[margin=1in]{geometry}
\usepackage{longtable,booktabs,array}
\usepackage{calc} % for calculating minipage widths
% Correct order of tables after \paragraph or \subparagraph
\usepackage{etoolbox}
\makeatletter
\patchcmd\longtable{\par}{\if@noskipsec\mbox{}\fi\par}{}{}
\makeatother
% Allow footnotes in longtable head/foot
\IfFileExists{footnotehyper.sty}{\usepackage{footnotehyper}}{\usepackage{footnote}}
\makesavenoteenv{longtable}
\usepackage{graphicx}
\makeatletter
\def\maxwidth{\ifdim\Gin@nat@width>\linewidth\linewidth\else\Gin@nat@width\fi}
\def\maxheight{\ifdim\Gin@nat@height>\textheight\textheight\else\Gin@nat@height\fi}
\makeatother
% Scale images if necessary, so that they will not overflow the page
% margins by default, and it is still possible to overwrite the defaults
% using explicit options in \includegraphics[width, height, ...]{}
\setkeys{Gin}{width=\maxwidth,height=\maxheight,keepaspectratio}
% Set default figure placement to htbp
\makeatletter
\def\fps@figure{htbp}
\makeatother
\setlength{\emergencystretch}{3em} % prevent overfull lines
\providecommand{\tightlist}{%
  \setlength{\itemsep}{0pt}\setlength{\parskip}{0pt}}
\setcounter{secnumdepth}{5}
\ifLuaTeX
  \usepackage{selnolig}  % disable illegal ligatures
\fi
\IfFileExists{bookmark.sty}{\usepackage{bookmark}}{\usepackage{hyperref}}
\IfFileExists{xurl.sty}{\usepackage{xurl}}{} % add URL line breaks if available
\urlstyle{same}
\hypersetup{
  pdftitle={CDM-Telematics Wiki Documentation},
  colorlinks=true,
  linkcolor={blue},
  filecolor={Maroon},
  citecolor={Blue},
  urlcolor={blue},
  pdfcreator={LaTeX via pandoc}}

\title{CDM-Telematics Wiki Documentation}
\author{}
\date{}

\begin{document}
\maketitle

{
\hypersetup{linkcolor=}
\setcounter{tocdepth}{3}
\tableofcontents
}
\setstretch{1.2}
\hypertarget{cdm-telematics-compiled-wiki-pages}{%
\section{CDM-Telematics compiled wiki
pages}\label{cdm-telematics-compiled-wiki-pages}}

\emph{Self-contained version with local images}

\hypertarget{version}{%
\subsection{Version}\label{version}}

This document was generated on 2025-11-07 16:16:58 UTC

\begin{longtable}[]{@{}
  >{\raggedright\arraybackslash}p{(\columnwidth - 0\tabcolsep) * \real{0.0556}}@{}}
\toprule\noalign{}
\begin{minipage}[b]{\linewidth}\raggedright
\# Operational entities
\end{minipage} \\
\midrule\noalign{}
\endhead
\bottomrule\noalign{}
\endlastfoot
\# Entity details \\
\#\# Purpose \\
Operational entities may all share time-independent properties
(attributes), which are described here. \\
\#\# Diagram \\
For the time being, a user-defined ``name'' is proposed. It is not
unique (one entity may have several names; different entities can have
the same name as the ``name'' is a vernacular, not an identifier. \\
\includegraphics{images/TAF_revisited_01a - Operational entity details.png} \\
\#\# Comments \\
The universally unique identifier is expected to be the entity IRI. IRIs
are the foundations of the Semantic Web. \\
\emph{Note: one reason for introducing the ``name'' property is that
annotation properties (such as rdfs:label) that would play this role are
left out of OWL to JSON transformers (CIM or PIM to PSM).} \\
Original page:
\href{https://github.com/UICrail/CDM-Telematics/wiki/01a-\%E2\%80\%90-Entity-details}{01a-‐-Entity-details.md} \\
\end{longtable}

\hypertarget{train-run}{%
\section{Train run}\label{train-run}}

\hypertarget{purpose}{%
\subsection{Purpose}\label{purpose}}

``Train run'' (commonly ``train'', in an operational context) designated
a train (characterized by a train number) running on a particular day.
It is what a dispatcher has to manage, not to be confused with: *
``train'' as a railway timetable entry; * ``train'' as a set of railway
vehicles, some of which are powered.

The expression ``train run'' was chosen because it best evokes the train
as a process.

The equivalent class in the ERA TAF ontology is
\href{http://data.europa.eu/949/Train}{Train}.

\hypertarget{diagram}{%
\subsection{Diagram}\label{diagram}}

\hypertarget{description}{%
\subsubsection{Description}\label{description}}

The GRAPHOL diagram is centered on the ``Train run'' class that has a
set of static properties (on the right) and time-dependent states (on
the left).

The static properties are directly derived from the Telematics TSIs and
ontology, e.g.~OTN (Operational Train Number).

The train states are further described in another diagram.

\begin{figure}
\centering
\includegraphics{images/TAF_revisited_02 - Train run.png}
\caption{train run diagram}
\end{figure}

\hypertarget{comments}{%
\subsubsection{Comments}\label{comments}}

\hypertarget{graphol-representation-of-properties}{%
\paragraph{GRAPHOL representation of
properties}\label{graphol-representation-of-properties}}

In GRAPHOL, object properties are represented by diamonds. Their name
often starts with ``has'', but this is a convention, not a rule.

Object properties are displayed as class A \textless-{[}white square{]}
- - {[}black square{]} -\textgreater{} class B, meaning P(A, B),
i.e.~individuals of A are related to individuals of B by a property
named P. Example: some Train (subject) has a train operating RU
(predicate) some Train operator (object).

Diamonds with a double-edged rim denote ``functional properties''. In
OWL2, a functional property is such that any subject will be associated
with at most one object. Inverse functional properties have a thick
black rim, and properties that are both functional and inverse
functional have a double-edged rim on one side and a thick black rim on
the other side.

When the property objects are data types (such as strings, integers,
timestamps\ldots), the property is called a ``data property'' in OWL2.
It is represented by a small circle, as is the case here with ``train
departure date''. The double rim also means functional {[}data{]}
property.

Note: in OWL2, object properties can be navigated in both directions, no
matter if the inverse property is defined (as is the case here for ``is
state of train run'') or not. Data properties are however one way (they
have no inverses). Of course the end user can \emph{query} OWL2 data to
find, e.g., ``all train runs occurring today'', but the logics
underpinning OWL2 ontologies (and executed by reasoners such as Pellet
or HermiT) cannot achieve that.

\hypertarget{owl2-subproperties}{%
\paragraph{OWL2 Subproperties}\label{owl2-subproperties}}

In OWL2, properties are first-class objects; there are subproperties as
there are subclasses. The semantics of ``sub'' is inclusion, and the
symbol for inclusion is the simple arrow in both cases. Explanation:

\begin{itemize}
\tightlist
\item
  for classes: class A is a subclass of class B iff (= if and only if)
  any individual of A is also an individual of B.
\item
  for properties: property P is a subproperty of property Q iff for any
  individuals X and Y satisfying P(X,Y), Q(X,Y) also holds.
\end{itemize}

The interest of defining subproperties is mainly a semantic one.
Example: ``empty mass'' or ``laden mass'' are subproperties of ``has
mass'', meaning that the expected value is a mass expressed in, say, kg,
but their meaning and relevance is quite precise.

In the present case, the subproperty arrow links the general-purpose
dul:isSettingFor property with its specialization, ``is state of train
run''.

\hypertarget{train-run-states-relate-to-exactly-one-train-run}{%
\paragraph{Train run states relate to exactly one train
run}\label{train-run-states-relate-to-exactly-one-train-run}}

This is partly expressed by property ``is state of train run'' being a
functional property. ``Functional'' implies 0 or 1 object.

A train run \emph{state} however is expected to describe exactly one
train run, as common sense dictates, not ``at most one''.

Enforcing such cardinality ``exactly one object'' is possible and is
done elsewhere, and in various ways (using OWL2 or SHACL constraints;
using an explicit cardinality or an existential condition). Here, we
chose to represent ``at most one'' by using a functional property, and
``at least one'' by stating an existential condition (``Train run
state'' is a subclass of all things X satisfying ``X is the state of
some train run''). The condition can be read from the diagram.

\hypertarget{a-train-run-involves-one-train}{%
\paragraph{A train run involves one
train}\label{a-train-run-involves-one-train}}

\ldots{} although its composition and loads may change between origin
and destination. See the Train-related wiki page.

\hypertarget{where-are-the-train-origin-and-destination}{%
\paragraph{Where are the train origin and
destination?}\label{where-are-the-train-origin-and-destination}}

The train destination may change in the course of the train run (think
of a re-routed train). This is why the train journey (planned, foreseen,
actual\ldots) are time-dependent, and documented using

\hypertarget{about-owl2-profiles}{%
\paragraph{About OWL2 profiles}\label{about-owl2-profiles}}

Explicit cardinalities (min=1, max=1) would look nicer, but are avoided
here for technical reasons. We try to stick to a subset of OWL2, namely
the OWL2 RL profile, to secure the tractability of queries in polynomial
time.

Original page:
\href{https://github.com/UICrail/CDM-Telematics/wiki/02-\%E2\%80\%90-Train-run}{02-‐-Train-run.md}

\begin{longtable}[]{@{}
  >{\raggedright\arraybackslash}p{(\columnwidth - 0\tabcolsep) * \real{0.0556}}@{}}
\toprule\noalign{}
\begin{minipage}[b]{\linewidth}\raggedright
\# Train servicing
\end{minipage} \\
\midrule\noalign{}
\endhead
\bottomrule\noalign{}
\endlastfoot
\# Operational Location \\
\#\# Purpose \\
Defining primary and subsidiary location codes. Locating fixed assets
(tracks, facilities) at the places designated by these codes. \\
\#\# Diagram \\
\includegraphics{images/TAF_revisited_03 - OperationalLocation.png} \\
\#\# Comments \\
\#\#\# Property ``at operational location'' \\
This property only applies to (= has range) spatially fixed assets such
as station tracks, facilities (yards, depots, stations, terminals).
There is no time information attached to it. \\
Please note that a yard \emph{has} a location; a yard \emph{is not} a
location. This confusion often occurs because a yard happens to be a
fixed thing. The confusion never occurs with moving things. In the
present ontology, fixed and moving things are treated homogeneously, so
fixed things are not assimilated to their location. \\
\#\#\# So where are moving things located at time t ? \\
The answer lies in the dul:Situation class and its subclasses (generally
called ``\ldots{} status'' in the present ontology). dul:Situation is
the class where time-dependent information is asserted. \\
Original page:
\href{https://github.com/UICrail/CDM-Telematics/wiki/03-\%E2\%80\%90-Operational-Location}{03-‐-Operational-Location.md} \\
\end{longtable}

\hypertarget{train}{%
\section{Train}\label{train}}

\hypertarget{purpose-1}{%
\subsection{Purpose}\label{purpose-1}}

The ``Train'' class designates the physical object (here:
dul:DesignedArtefact) that is moved in the process of running a train,
the ``Train run''. That includes not only the rolling stock (considered
in running order, with all supplies included), but also the loaded
freight and (with a minus sign) the spent fuel or sand.

\hypertarget{diagram-1}{%
\subsubsection{Diagram}\label{diagram-1}}

The composition of the Train varies between Journey Segments (where
wagons can be added or removed, locomotives can be changed, etc.). We
choose to consider that for each particular train run, we have a single
train (i.e.~a single individual of class Train) with variable
composition and loads. This choice best coincides with the way people
speak.

\begin{figure}
\centering
\includegraphics{images/TAF_revisited_04 - Train.png}
\caption{Train}
\end{figure}

Original page:
\href{https://github.com/UICrail/CDM-Telematics/wiki/04-\%E2\%80\%90-Train}{04-‐-Train.md}

\begin{longtable}[]{@{}
  >{\raggedright\arraybackslash}p{(\columnwidth - 0\tabcolsep) * \real{0.0556}}@{}}
\toprule\noalign{}
\begin{minipage}[b]{\linewidth}\raggedright
\# Wagon
\end{minipage} \\
\midrule\noalign{}
\endhead
\bottomrule\noalign{}
\endlastfoot
\# Intermodal Transport Unit \\
\#\# Purpose \\
\#\# Diagram \\
\includegraphics{images/TAF_revisited_06 - ITU.png} \\
\#\# Comments \\
Original page:
\href{https://github.com/UICrail/CDM-Telematics/wiki/06-\%E2\%80\%90-Intermodal-Transport-Unit}{06-‐-Intermodal-Transport-Unit.md} \\
\end{longtable}

\hypertarget{cargo}{%
\section{Cargo}\label{cargo}}

\hypertarget{purpose-2}{%
\subsection{Purpose}\label{purpose-2}}

\hypertarget{diagram-2}{%
\subsection{Diagram}\label{diagram-2}}

\hypertarget{comments-1}{%
\subsection{Comments}\label{comments-1}}

Original page:
\href{https://github.com/UICrail/CDM-Telematics/wiki/06a-\%E2\%80\%90-Cargo}{06a-‐-Cargo.md}

\begin{longtable}[]{@{}
  >{\raggedright\arraybackslash}p{(\columnwidth - 0\tabcolsep) * \real{0.0556}}@{}}
\toprule\noalign{}
\begin{minipage}[b]{\linewidth}\raggedright
\# Track
\end{minipage} \\
\midrule\noalign{}
\endhead
\bottomrule\noalign{}
\endlastfoot
\# Facility \\
\#\# Purpose \\
\#\# Diagram \\
\includegraphics{images/TAF_revisited_08 - Facility.png} \\
\#\# Comments \\
Original page:
\href{https://github.com/UICrail/CDM-Telematics/wiki/08-\%E2\%80\%90-Facility}{08-‐-Facility.md} \\
\end{longtable}

\hypertarget{traction-role}{%
\section{Traction role}\label{traction-role}}

\hypertarget{purpose-3}{%
\subsection{Purpose}\label{purpose-3}}

\hypertarget{diagram-3}{%
\subsection{Diagram}\label{diagram-3}}

\begin{figure}
\centering
\includegraphics{images/TAF_revisited_09 - Traction role.png}
\caption{Traction role}
\end{figure}

\hypertarget{comments-2}{%
\subsection{Comments}\label{comments-2}}

Original page:
\href{https://github.com/UICrail/CDM-Telematics/wiki/09-\%E2\%80\%90-Traction-role}{09-‐-Traction-role.md}

\begin{longtable}[]{@{}
  >{\raggedright\arraybackslash}p{(\columnwidth - 0\tabcolsep) * \real{0.0556}}@{}}
\toprule\noalign{}
\begin{minipage}[b]{\linewidth}\raggedright
\# Load Role
\end{minipage} \\
\midrule\noalign{}
\endhead
\bottomrule\noalign{}
\endlastfoot
\# Operational roles \\
\#\# Purpose \\
Companies may play different operational roles at different times.
Examples of operational roles are: Train operator, Cargo carrier, Lead
RU, Lead carrier (the latter is encountered in the TAF TSI). \\
\#\# Diagram \\
Operational roles are currently listed as members of class
OperationalRole. In the future, these may be replaced by a SKOS concept
scheme provided by ERA. \\
The answer to ``who plays the role'' is outside this ontology. Such
companies are by definition subclasses of dul:Organization and
regorg:RegisteredOrganization. \\
dul:Organization tells that its members are able to play a role.
regorg:Organization provides lots of useful attributes (legal name,
jurisdiction, registration authority, etc.). The W3C
\href{https://www.w3.org/TR/vocab-regorg/}{REGORG ontology} is based on
the W3C ORG ontology and is recommended for use by the EC (see
\href{https://interoperable-europe.ec.europa.eu/collection/registered-organization-vocabulary}{this
ontology collection item} and
\href{https://interoperable-europe.ec.europa.eu/collection/semic-support-centre/news/w3c-publishes-two-specificati}{this
announcement page} from the Interoperable Europe Portal. \\
\includegraphics{images/TAF_revisited_11 - Operational role.png} \\
\#\# Comments \\
\#\#\# No dedicated class for organizations - just a blank node \\
We simply use a ``blank node'' (a class without a name) that is the
intersection (``and'') of dul:Organization and
rerorg:RegisteredOrganization. \\
What the diagram expresses (and what OWL2 says) is that anything that
plays an operational role in the context of telematics is, by
definition, both a dul:Organization (hence not a person) and a
rerorg:Organization (with some or all of the foreseen properties
provided by RERORG). \\
Original page:
\href{https://github.com/UICrail/CDM-Telematics/wiki/11-\%E2\%80\%90-Operational-roles}{11-‐-Operational-roles.md} \\
\end{longtable}

\hypertarget{versioned-description}{%
\section{Versioned description}\label{versioned-description}}

\hypertarget{purpose-4}{%
\subsection{Purpose}\label{purpose-4}}

\hypertarget{diagram-4}{%
\subsection{Diagram}\label{diagram-4}}

\begin{figure}
\centering
\includegraphics{images/TAF_revisited_12 - Versioned Description.png}
\caption{Versioned description}
\end{figure}

\hypertarget{comments-3}{%
\subsection{Comments}\label{comments-3}}

Original page:
\href{https://github.com/UICrail/CDM-Telematics/wiki/12-\%E2\%80\%90-Versioned-description}{12-‐-Versioned-description.md}

\begin{longtable}[]{@{}
  >{\raggedright\arraybackslash}p{(\columnwidth - 0\tabcolsep) * \real{0.0556}}@{}}
\toprule\noalign{}
\begin{minipage}[b]{\linewidth}\raggedright
\# Journey
\end{minipage} \\
\midrule\noalign{}
\endhead
\bottomrule\noalign{}
\endlastfoot
\# Journey Schedule \\
\#\# Purpose \\
Represent the train journey (planned or executed) and possibly the train
path in one uniform way. \\
The Journey may be composed of a sequence of journey sections, each
having an origin and a destination (operational locations). Obviously,
the destination of section N is expected to also be the origin of
section N+1, and times must be increasing (except when midnight is
passed). \\
\#\# Diagram \\
\includegraphics{images/TAF_revisited_12b - Journey Schedule.png} \\
\#\# Comments \\
\#\#\# Nested Lists \\
Since any Journey section is a Journey schedule, it can be in turn
broken down. This is convenient, as it allows to insert pass-through
locations at a later stage when deemed convenient. \\
\#\#\# Not a speed profile \\
Locations documented by Journey Schedule are, for the time being, only
Operational Locations. The Journey Schedule is not suitable for
representing a speed profile with temporary speed restrictions for
instance. \\
\#\#\# Data consistency \\
It is a user responsibility to check the order and consistency of
journey sections. The ontology will preserve the sequence (using a List
ontology, because OWL2 has no primitive concepts for order), but whether
the sequence \emph{makes sense} must be checked by other means, such as
SWRL rules, or queries (possibly embedded in SHACL), etc. \\
\#\#\# List ontology \\
The List ontology used here is different from the one used in IfcOwl
(the ontology version of Industry Foundation Classes). Later on, one of
the two ontologies may be eliminated in favor of the other. \\
Original page:
\href{https://github.com/UICrail/CDM-Telematics/wiki/12b-\%E2\%80\%90-Journey-Schedule}{12b-‐-Journey-Schedule.md} \\
\end{longtable}

\hypertarget{journey-schedule-properties}{%
\section{Journey Schedule
properties}\label{journey-schedule-properties}}

\hypertarget{purpose-5}{%
\subsection{Purpose}\label{purpose-5}}

Properties are made available by the List ontology that allow to
``chain'' and navigate the Journey Sections. Some are illustrated here.

\hypertarget{diagram-5}{%
\subsection{Diagram}\label{diagram-5}}

Journey sections are chained with no branches and no loops. This is
expressed by having properties (next section, previous section) that are
both functional and inverse functional: the diamonds have both a double
rim (functional, multiplicity 0..1) and a thick black rim (inverse
functional, multiplicity 0..1).

\begin{figure}
\centering
\includegraphics{images/TAF_revisited_12c - Journey Schedule Properties.png}
\caption{Journey schedule properties}
\end{figure}

On the left, you see the GRAPHOL representation of the OWL2 assertion:
``in journey schedule'' is the exact equivalent (double arrow!) of the
inverse of ``has journey section''.

\emph{Note: since ``in journey schedule'' is the inverse of an inverse
functional property, it consequently is a functional property. The
diagram does not declare it (simple rim instead of double rim) but OWL2
reasoners will infer it.}

\hypertarget{comments-4}{%
\subsection{Comments}\label{comments-4}}

(about the identification of the first and last item)

Original page:
\href{https://github.com/UICrail/CDM-Telematics/wiki/12c-\%E2\%80\%90-Journey-Schedule-properties}{12c-‐-Journey-Schedule-properties.md}

\begin{longtable}[]{@{}
  >{\raggedright\arraybackslash}p{(\columnwidth - 0\tabcolsep) * \real{0.0556}}@{}}
\toprule\noalign{}
\begin{minipage}[b]{\linewidth}\raggedright
\# Operational State
\end{minipage} \\
\midrule\noalign{}
\endhead
\bottomrule\noalign{}
\endlastfoot
\# Train run state \\
\#\# Purpose \\
\#\# Diagram \\
\includegraphics{images/TAF_revisited_13a - Train run state.png} \\
\#\# Comments \\
Original page:
\href{https://github.com/UICrail/CDM-Telematics/wiki/13a-\%E2\%80\%90-Train-run-state}{13a-‐-Train-run-state.md} \\
\end{longtable}

\hypertarget{train-state}{%
\section{Train state}\label{train-state}}

\hypertarget{purpose-6}{%
\subsection{Purpose}\label{purpose-6}}

\hypertarget{diagram-6}{%
\subsection{Diagram}\label{diagram-6}}

\begin{figure}
\centering
\includegraphics{images/TAF_revisited_13b - Train state.png}
\caption{Train state}
\end{figure}

\hypertarget{comments-5}{%
\subsection{Comments}\label{comments-5}}

Original page:
\href{https://github.com/UICrail/CDM-Telematics/wiki/13b-\%E2\%80\%90-Train-state}{13b-‐-Train-state.md}

\begin{longtable}[]{@{}
  >{\raggedright\arraybackslash}p{(\columnwidth - 0\tabcolsep) * \real{0.0556}}@{}}
\toprule\noalign{}
\begin{minipage}[b]{\linewidth}\raggedright
\# Load State
\end{minipage} \\
\midrule\noalign{}
\endhead
\bottomrule\noalign{}
\endlastfoot
\# Message \\
\#\# Purpose \\
\#\# Diagram \\
\includegraphics{images/TAF_revisited_14 - Message.png} \\
\#\# Comments \\
Original page:
\href{https://github.com/UICrail/CDM-Telematics/wiki/14-\%E2\%80\%90-Message}{14-‐-Message.md} \\
\end{longtable}

\hypertarget{image}{%
\section{Image}\label{image}}

\hypertarget{purpose-7}{%
\subsection{Purpose}\label{purpose-7}}

Provide a reference to images, and access image contents (e.g.~a PNG
file).

\hypertarget{diagram-7}{%
\subsection{Diagram}\label{diagram-7}}

The image reference (URI) is separated from the contents (Content
location, identified by its own URI). It is likely that they will not
share the same workspace. Persistence and backup mechanisms are likely
to be very different: e.g.~images will require frequent archiving, owing
to their ``weight''.

The content URI may be a URL (dereferenceable) or another type of URI if
only a ``unique key'' in a specific database is needed.

\begin{figure}
\centering
\includegraphics{images/TAF_revisited_15 - Image.png}
\caption{image class}
\end{figure}

\hypertarget{comments-6}{%
\subsection{Comments}\label{comments-6}}

\hypertarget{metadata}{%
\subsubsection{Metadata}\label{metadata}}

Image metadata persistence (provenance\ldots) is of course crucial and
shall be discussed in due time.

Original page:
\href{https://github.com/UICrail/CDM-Telematics/wiki/15-\%E2\%80\%90-Image}{15-‐-Image.md}

\begin{longtable}[]{@{}
  >{\raggedright\arraybackslash}p{(\columnwidth - 0\tabcolsep) * \real{0.0556}}@{}}
\toprule\noalign{}
\begin{minipage}[b]{\linewidth}\raggedright
\# RID codes
\end{minipage} \\
\midrule\noalign{}
\endhead
\bottomrule\noalign{}
\endlastfoot
\# Time \\
\#\# Purpose \\
Propose ``time entities'' to be consumed by time-dependent entities
(such as ``train speed'' or ``custom clearance status''). \\
The time entities are derived from the
\href{https://www.w3.org/TR/owl-time/}{W3C time ontology} and aligned
with the DUL TimeInterval concept. \\
\#\# Diagram \\
An operational time is either an instant or an interval: these are
disjoint classes, as the black flattened hexagon indicates in the
diagram. The distinction between instant and interval rests on
semantics, not on timestamp values: an interval has a beginning and an
end (whether or not the values are known), while an instant has a
beginning (instant) and an end (instant). These may happen to coincide,
i.e.~have equal timestamp values. \\
The GRAPHOL diagram expresses that each interval is expected to have
exactly one beginning and one end, by means of OWL universal
restrictions (``forAll''), and that the beginning or end are of type
``Operational instant''. \\
\emph{Note: if the data provide two beginnings for an interval, the
logical consequence is that the beginnings B1 and B2 are the same
individual (:B1 owl:sameAs :B2) and any OWL reasoner will infer that and
inform the user accordingly. Then the user software may want to compare
the respective timestamp values (OWL2 ignores them, so use SPARQL or
code). If timestamps of B1 and B2 fail the test for equality (say,
difference is more than one millisecond? which is context-dependent),
then the user has detected an inconsistency and should consider
resolving it. In a world of imperfect data, wrong data should not break
the database. Ontologies, being robust against such data errors, are a
suitable tool to represent them, but the onus of error detection is
partly on the user: OWL2 has no numeric operators.} \\
Operational times (instants or intervals) may have a ``date and time of
issue''. This non-mandatory information is of interest in the case of
repeated forecasting or revision exercises. The bulk of exchanged times,
in an operational environment, is composed forecasts or revisions, so
this ``time property of time'' is everything but ludicrous. \\
\includegraphics{images/TAF_revisited_90 - Time.png} \\
The bottom part (Temporal role class and individuals) add meaning to
time-dependent situations, i.e.~whether they relate to a planned,
revised, forecast, actual, etc. situation (the DUL term) or state (our
used term). \\
\#\# Comments \\
\#\#\# Real Time applications \\
The time representation is conceptually compatible with the
\href{https://www.omg.org/omgmarte/}{UML MARTE profile}, so it can serve
as a basis for a demanding real-time application. The main missing
concept is that of Clock: we currently assume all clocks to be
synchronized, so we ignore them altogether (there is no need for a Clock
class). \\
We do not distinguish intervals from ``proper intervals'' (length
\textgreater{} 0) in the sense of the W3C Time ontology: as no
assumptions are made regarding the timestamps, if may well happen that
the beginning and the end of an interval are distinct (in actual time),
but have the same timestamp value. Calling it a ``proper interval''
would then be a semantic error. \\
\#\#\# Against open-ended intervals \\
Leaving out the beginning or end of an interval \emph{may} be
interpreted as the interval being ``open-ended''. We would like to
sternly warn against such convention, since it breaks the monotonicity
of time reasoning. Example: if a fire extinguisher has a usability
interval but the end is missing from the available data, it shall not be
understood as ``usable forever'', as this conclusion would immediately
be falsified by the provision, at some stage, of the missing
information. \\
In the context of railway applications, nothing is actually open-ended
in time, and most data may be made available with some delay, so the
``open ended'' interpretation of any interval with missing data would be
a gross mistake. \\
Original page:
\href{https://github.com/UICrail/CDM-Telematics/wiki/90-\%E2\%80\%90-Time}{90-‐-Time.md} \\
\end{longtable}

\hypertarget{varia}{%
\section{Varia}\label{varia}}

\hypertarget{purpose-8}{%
\subsection{Purpose}\label{purpose-8}}

\hypertarget{diagram-8}{%
\subsection{Diagram}\label{diagram-8}}

\begin{figure}
\centering
\includegraphics{images/TAF_revisited_99 - Varia.png}
\caption{Varia}
\end{figure}

\hypertarget{comments-7}{%
\subsection{Comments}\label{comments-7}}

Original page:
\href{https://github.com/UICrail/CDM-Telematics/wiki/95-\%E2\%80\%90-Varia}{95-‐-Varia.md}

\begin{longtable}[]{@{}
  >{\raggedright\arraybackslash}p{(\columnwidth - 0\tabcolsep) * \real{0.0556}}@{}}
\toprule\noalign{}
\endhead
\bottomrule\noalign{}
\endlastfoot
\# Dependencies \\
\#\# General and upper ontologies \\
\#\#\# W3C Time ontology \\
\#\#\# SOSA/SSN \\
\#\#\# QUDT \\
\#\#\# DOLCE+DnS Ultralite \\
\#\#\# REGORG, ORG \\
\#\# Semantic RSM ontologies \\
\#\#\# Rolling stock ontologies \\
\#\#\#\# Consist \\
\#\#\#\# Typology \\
\#\# ERA Concept Schemes \\
Original page:
\href{https://github.com/UICrail/CDM-Telematics/wiki/97-\%E2\%80\%90-Dependencies}{97-‐-Dependencies.md} \\
\end{longtable}

\hypertarget{references}{%
\section{References}\label{references}}

\ldots{}

Original page:
\href{https://github.com/UICrail/CDM-Telematics/wiki/99-\%E2\%80\%90-References}{99-‐-References.md}

\end{document}
