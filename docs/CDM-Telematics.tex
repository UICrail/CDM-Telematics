% Options for packages loaded elsewhere
\PassOptionsToPackage{unicode}{hyperref}
\PassOptionsToPackage{hyphens}{url}
%
\documentclass[
]{article}
\usepackage{amsmath,amssymb}
\usepackage{iftex}
\ifPDFTeX
  \usepackage[T1]{fontenc}
  \usepackage[utf8]{inputenc}
  \usepackage{textcomp} % provide euro and other symbols
\else % if luatex or xetex
  \usepackage{unicode-math} % this also loads fontspec
  \defaultfontfeatures{Scale=MatchLowercase}
  \defaultfontfeatures[\rmfamily]{Ligatures=TeX,Scale=1}
\fi
\usepackage{lmodern}
\ifPDFTeX\else
  % xetex/luatex font selection
\fi
% Use upquote if available, for straight quotes in verbatim environments
\IfFileExists{upquote.sty}{\usepackage{upquote}}{}
\IfFileExists{microtype.sty}{% use microtype if available
  \usepackage[]{microtype}
  \UseMicrotypeSet[protrusion]{basicmath} % disable protrusion for tt fonts
}{}
\makeatletter
\@ifundefined{KOMAClassName}{% if non-KOMA class
  \IfFileExists{parskip.sty}{%
    \usepackage{parskip}
  }{% else
    \setlength{\parindent}{0pt}
    \setlength{\parskip}{6pt plus 2pt minus 1pt}}
}{% if KOMA class
  \KOMAoptions{parskip=half}}
\makeatother
\usepackage{xcolor}
\usepackage{longtable,booktabs,array}
\usepackage{calc} % for calculating minipage widths
% Correct order of tables after \paragraph or \subparagraph
\usepackage{etoolbox}
\makeatletter
\patchcmd\longtable{\par}{\if@noskipsec\mbox{}\fi\par}{}{}
\makeatother
% Allow footnotes in longtable head/foot
\IfFileExists{footnotehyper.sty}{\usepackage{footnotehyper}}{\usepackage{footnote}}
\makesavenoteenv{longtable}
\usepackage{graphicx}
\makeatletter
\def\maxwidth{\ifdim\Gin@nat@width>\linewidth\linewidth\else\Gin@nat@width\fi}
\def\maxheight{\ifdim\Gin@nat@height>\textheight\textheight\else\Gin@nat@height\fi}
\makeatother
% Scale images if necessary, so that they will not overflow the page
% margins by default, and it is still possible to overwrite the defaults
% using explicit options in \includegraphics[width, height, ...]{}
\setkeys{Gin}{width=\maxwidth,height=\maxheight,keepaspectratio}
% Set default figure placement to htbp
\makeatletter
\def\fps@figure{htbp}
\makeatother
\setlength{\emergencystretch}{3em} % prevent overfull lines
\providecommand{\tightlist}{%
  \setlength{\itemsep}{0pt}\setlength{\parskip}{0pt}}
\setcounter{secnumdepth}{5}
\ifLuaTeX
  \usepackage{selnolig}  % disable illegal ligatures
\fi
\IfFileExists{bookmark.sty}{\usepackage{bookmark}}{\usepackage{hyperref}}
\IfFileExists{xurl.sty}{\usepackage{xurl}}{} % add URL line breaks if available
\urlstyle{same}
\hypersetup{
  pdftitle={CDM-Telematics Wiki Documentation},
  hidelinks,
  pdfcreator={LaTeX via pandoc}}

\title{CDM-Telematics Wiki Documentation}
\author{}
\date{}

\begin{document}
\maketitle

{
\setcounter{tocdepth}{3}
\tableofcontents
}
\hypertarget{cdm-telematics-compiled-wiki-pages}{%
\section{CDM-Telematics compiled wiki
pages}\label{cdm-telematics-compiled-wiki-pages}}

\emph{Self-contained version with local images}

\hypertarget{version}{%
\subsection{Version}\label{version}}

This document was generated on 2025-11-07 15:45:32 UTC

\begin{longtable}[]{@{}
  >{\raggedright\arraybackslash}p{(\columnwidth - 0\tabcolsep) * \real{0.0556}}@{}}
\toprule\noalign{}
\begin{minipage}[b]{\linewidth}\raggedright
\#\# Contents
\end{minipage} \\
\midrule\noalign{}
\endhead
\bottomrule\noalign{}
\endlastfoot
\# Operational entities \\
\#\# Purpose \\
This diagram contains the main classes used for describing operations.
Their scope is not necessarily restricted to the TAF TSI scope: for
instance, containers (assets) and their ``freight load'' role (when
loaded on a wagon) are described. \\
\#\# Presentation \\
The left part (colored classes) shows the DUL (DOLCE+DnS Ultralite)
concepts underpinning the model. The rest represents the taxonomy that
is specific to the CDM-TAF ontology (with top concepts to the right). \\
Next to the right of the DUL classes are the classes describing our
domain (operations). Their names should sound familiar. \\
Then, further right, are the superclasses (grouping our domain classes).
They will mostly be ignored by users. \\
Details are provided in subsequent pages. \\
\#\# About ``DOLCE + DnS Ultralite'' (DUL) \\
\href{https://akswnc7.informatik.uni-leipzig.de/dstreitmatter/archivo/ontologydesignpatterns.org/ont--dul--DUL--owl/2021.06.07-182648/ont--dul--DUL--owl_type=pyLodeDoc.html}{DUL}
is an ``upper ontology'' defining very general concepts (physical
object, role, description\ldots). It is suitable for describing domains
mixing assets, processes, and all sorts of ``social constructions'', all
of them possibly time-dependent. \\
End users can safely ignore this upper ontology. It is however useful
for those who wish to understand, maintain, or extend the CDM-TAF
ontology, since it contributes to separation of concerns and consistent
design patterns. \\
\includegraphics{images/TAF_revisited_01 - OperationalEntities.png} \\
\#\# Comments on the diagram \\
\#\#\# GRAPHOL conventions \\
GRAPHOL is able to represent all OWL2 elements and relationships
graphically. The graphs may evoke E/R or UML diagrams, but there are
significant differences resulting from the very principles of
ontologies. \\
\#\#\#\# Classes and subclasses \\
Square boxes represent classes (sets of individuals). \\
The graph A -\textgreater{} B means ``A is included in B''. When A and B
are classes (which is the case here), this means ``A is a subclass of
B''. \\
Keep in mind that ontology classes are sets, rather than types (as would
generally be the case in object-oriented programming). All set
operations, such as intersection, union, or complement, hence apply to
ontology classes. \\
\#\#\#\# Disjoint unions \\
A black, flattened hexagon means ``disjoint union of'' whatever classes
are attached to the hexagon by dashed lines. \\
For instance, Train, Wagon, Intermodal Transport Unit, Track, and
Facility, are bundled into an anonymous ``disjoint union''. This implies
that e.g.~a Train cannot at the same time be a Wagon, etc. \\
This anonymous, disjoint union is a subclass of ``Operational artifact''
(did you notice the direction of the arrow?). This means that
``operational artifact'' contains (or includes) the mutually disjoint
Train, Wagon, etc. classes, \emph{and possibly could include other
classes not mentioned here}, because out of scope. \\
Original page:
\href{https://github.com/UICrail/CDM-Telematics/wiki/01-\%E2\%80\%90-Operational-entities}{01-‐-Operational-entities.md} \\
\end{longtable}

\hypertarget{entity-details}{%
\section{Entity details}\label{entity-details}}

\hypertarget{purpose}{%
\subsection{Purpose}\label{purpose}}

Operational entities may all share time-independent properties
(attributes), which are described here.

\hypertarget{diagram}{%
\subsection{Diagram}\label{diagram}}

For the time being, a user-defined ``name'' is proposed. It is not
unique (one entity may have several names; different entities can have
the same name as the ``name'' is a vernacular, not an identifier.

\begin{figure}
\centering
\includegraphics{images/TAF_revisited_01a - Operational entity details.png}
\caption{Operational entity details}
\end{figure}

\hypertarget{comments}{%
\subsection{Comments}\label{comments}}

The universally unique identifier is expected to be the entity IRI. IRIs
are the foundations of the Semantic Web.

\emph{Note: one reason for introducing the ``name'' property is that
annotation properties (such as rdfs:label) that would play this role are
left out of OWL to JSON transformers (CIM or PIM to PSM).}

Original page:
\href{https://github.com/UICrail/CDM-Telematics/wiki/01a-\%E2\%80\%90-Entity-details}{01a-‐-Entity-details.md}

\begin{longtable}[]{@{}
  >{\raggedright\arraybackslash}p{(\columnwidth - 0\tabcolsep) * \real{0.0556}}@{}}
\toprule\noalign{}
\begin{minipage}[b]{\linewidth}\raggedright
\# Train run
\end{minipage} \\
\midrule\noalign{}
\endhead
\bottomrule\noalign{}
\endlastfoot
\# Train servicing \\
\#\# Purpose \\
\#\# Diagram \\
\includegraphics{images/TAF_revisited_02a - Train Servicing.png} \\
\#\# Comments \\
Original page:
\href{https://github.com/UICrail/CDM-Telematics/wiki/02a-\%E2\%80\%90-Train-servicing}{02a-‐-Train-servicing.md} \\
\end{longtable}

\hypertarget{operational-location}{%
\section{Operational Location}\label{operational-location}}

\hypertarget{purpose-1}{%
\subsection{Purpose}\label{purpose-1}}

Defining primary and subsidiary location codes. Locating fixed assets
(tracks, facilities) at the places designated by these codes.

\hypertarget{diagram-1}{%
\subsection{Diagram}\label{diagram-1}}

\begin{figure}
\centering
\includegraphics{images/TAF_revisited_03 - OperationalLocation.png}
\caption{operational location}
\end{figure}

\hypertarget{comments-1}{%
\subsection{Comments}\label{comments-1}}

\hypertarget{property-at-operational-location}{%
\subsubsection{Property ``at operational
location''}\label{property-at-operational-location}}

This property only applies to (= has range) spatially fixed assets such
as station tracks, facilities (yards, depots, stations, terminals).
There is no time information attached to it.

Please note that a yard \emph{has} a location; a yard \emph{is not} a
location. This confusion often occurs because a yard happens to be a
fixed thing. The confusion never occurs with moving things. In the
present ontology, fixed and moving things are treated homogeneously, so
fixed things are not assimilated to their location.

\hypertarget{so-where-are-moving-things-located-at-time-t}{%
\subsubsection{So where are moving things located at time t
?}\label{so-where-are-moving-things-located-at-time-t}}

The answer lies in the dul:Situation class and its subclasses (generally
called ``\ldots{} status'' in the present ontology). dul:Situation is
the class where time-dependent information is asserted.

Original page:
\href{https://github.com/UICrail/CDM-Telematics/wiki/03-\%E2\%80\%90-Operational-Location}{03-‐-Operational-Location.md}

\begin{longtable}[]{@{}
  >{\raggedright\arraybackslash}p{(\columnwidth - 0\tabcolsep) * \real{0.0556}}@{}}
\toprule\noalign{}
\begin{minipage}[b]{\linewidth}\raggedright
\# Train
\end{minipage} \\
\midrule\noalign{}
\endhead
\bottomrule\noalign{}
\endlastfoot
\# Wagon \\
\#\# Purpose \\
\#\# Diagram \\
\includegraphics{images/TAF_revisited_05 - Wagon.png} \\
\#\# Comments \\
Original page:
\href{https://github.com/UICrail/CDM-Telematics/wiki/05-\%E2\%80\%90-Wagon}{05-‐-Wagon.md} \\
\end{longtable}

\hypertarget{intermodal-transport-unit}{%
\section{Intermodal Transport Unit}\label{intermodal-transport-unit}}

\hypertarget{purpose-2}{%
\subsection{Purpose}\label{purpose-2}}

\hypertarget{diagram-2}{%
\subsection{Diagram}\label{diagram-2}}

\begin{figure}
\centering
\includegraphics{images/TAF_revisited_06 - ITU.png}
\caption{ITU}
\end{figure}

\hypertarget{comments-2}{%
\subsection{Comments}\label{comments-2}}

Original page:
\href{https://github.com/UICrail/CDM-Telematics/wiki/06-\%E2\%80\%90-Intermodal-Transport-Unit}{06-‐-Intermodal-Transport-Unit.md}

\begin{longtable}[]{@{}
  >{\raggedright\arraybackslash}p{(\columnwidth - 0\tabcolsep) * \real{0.0556}}@{}}
\toprule\noalign{}
\begin{minipage}[b]{\linewidth}\raggedright
\# Cargo
\end{minipage} \\
\midrule\noalign{}
\endhead
\bottomrule\noalign{}
\endlastfoot
\# Track \\
\#\# Purpose \\
\#\# Diagram \\
\includegraphics{images/TAF_revisited_07 - Track.png} \\
\#\# Comments \\
Original page:
\href{https://github.com/UICrail/CDM-Telematics/wiki/07-\%E2\%80\%90-Track}{07-‐-Track.md} \\
\end{longtable}

\hypertarget{facility}{%
\section{Facility}\label{facility}}

\hypertarget{purpose-3}{%
\subsection{Purpose}\label{purpose-3}}

\hypertarget{diagram-3}{%
\subsection{Diagram}\label{diagram-3}}

\begin{figure}
\centering
\includegraphics{images/TAF_revisited_08 - Facility.png}
\caption{Facility}
\end{figure}

\hypertarget{comments-3}{%
\subsection{Comments}\label{comments-3}}

Original page:
\href{https://github.com/UICrail/CDM-Telematics/wiki/08-\%E2\%80\%90-Facility}{08-‐-Facility.md}

\begin{longtable}[]{@{}
  >{\raggedright\arraybackslash}p{(\columnwidth - 0\tabcolsep) * \real{0.0556}}@{}}
\toprule\noalign{}
\begin{minipage}[b]{\linewidth}\raggedright
\# Traction role
\end{minipage} \\
\midrule\noalign{}
\endhead
\bottomrule\noalign{}
\endlastfoot
\# Load Role \\
\#\# Purpose \\
\#\# Diagram \\
\includegraphics{images/TAF_revisited_10 - Load Role.png} \\
\#\# Comments \\
Original page:
\href{https://github.com/UICrail/CDM-Telematics/wiki/10-\%E2\%80\%90-Load-Role}{10-‐-Load-Role.md} \\
\end{longtable}

\hypertarget{operational-roles}{%
\section{Operational roles}\label{operational-roles}}

\hypertarget{purpose-4}{%
\subsection{Purpose}\label{purpose-4}}

Companies may play different operational roles at different times.
Examples of operational roles are: Train operator, Cargo carrier, Lead
RU, Lead carrier (the latter is encountered in the TAF TSI).

\hypertarget{diagram-4}{%
\subsection{Diagram}\label{diagram-4}}

Operational roles are currently listed as members of class
OperationalRole. In the future, these may be replaced by a SKOS concept
scheme provided by ERA.

The answer to ``who plays the role'' is outside this ontology. Such
companies are by definition subclasses of dul:Organization and
regorg:RegisteredOrganization.

dul:Organization tells that its members are able to play a role.
regorg:Organization provides lots of useful attributes (legal name,
jurisdiction, registration authority, etc.). The W3C
\href{https://www.w3.org/TR/vocab-regorg/}{REGORG ontology} is based on
the W3C ORG ontology and is recommended for use by the EC (see
\href{https://interoperable-europe.ec.europa.eu/collection/registered-organization-vocabulary}{this
ontology collection item} and
\href{https://interoperable-europe.ec.europa.eu/collection/semic-support-centre/news/w3c-publishes-two-specificati}{this
announcement page} from the Interoperable Europe Portal.

\begin{figure}
\centering
\includegraphics{images/TAF_revisited_11 - Operational role.png}
\caption{operational roles}
\end{figure}

\hypertarget{comments-4}{%
\subsection{Comments}\label{comments-4}}

\hypertarget{no-dedicated-class-for-organizations---just-a-blank-node}{%
\subsubsection{No dedicated class for organizations - just a blank
node}\label{no-dedicated-class-for-organizations---just-a-blank-node}}

We simply use a ``blank node'' (a class without a name) that is the
intersection (``and'') of dul:Organization and
rerorg:RegisteredOrganization.

What the diagram expresses (and what OWL2 says) is that anything that
plays an operational role in the context of telematics is, by
definition, both a dul:Organization (hence not a person) and a
rerorg:Organization (with some or all of the foreseen properties
provided by RERORG).

Original page:
\href{https://github.com/UICrail/CDM-Telematics/wiki/11-\%E2\%80\%90-Operational-roles}{11-‐-Operational-roles.md}

\begin{longtable}[]{@{}
  >{\raggedright\arraybackslash}p{(\columnwidth - 0\tabcolsep) * \real{0.0556}}@{}}
\toprule\noalign{}
\begin{minipage}[b]{\linewidth}\raggedright
\# Versioned description
\end{minipage} \\
\midrule\noalign{}
\endhead
\bottomrule\noalign{}
\endlastfoot
\# Journey \\
\#\# Purpose \\
\#\# Diagram \\
\includegraphics{images/TAF_revisited_12a - Journey.png} \\
\#\# Comments \\
Original page:
\href{https://github.com/UICrail/CDM-Telematics/wiki/12a-\%E2\%80\%90-Journey}{12a-‐-Journey.md} \\
\end{longtable}

\hypertarget{journey-schedule}{%
\section{Journey Schedule}\label{journey-schedule}}

\hypertarget{purpose-5}{%
\subsection{Purpose}\label{purpose-5}}

Represent the train journey (planned or executed) and possibly the train
path in one uniform way.

The Journey may be composed of a sequence of journey sections, each
having an origin and a destination (operational locations). Obviously,
the destination of section N is expected to also be the origin of
section N+1, and times must be increasing (except when midnight is
passed).

\hypertarget{diagram-5}{%
\subsection{Diagram}\label{diagram-5}}

\begin{figure}
\centering
\includegraphics{images/TAF_revisited_12b - Journey Schedule.png}
\caption{Journey Schedule}
\end{figure}

\hypertarget{comments-5}{%
\subsection{Comments}\label{comments-5}}

\hypertarget{nested-lists}{%
\subsubsection{Nested Lists}\label{nested-lists}}

Since any Journey section is a Journey schedule, it can be in turn
broken down. This is convenient, as it allows to insert pass-through
locations at a later stage when deemed convenient.

\hypertarget{not-a-speed-profile}{%
\subsubsection{Not a speed profile}\label{not-a-speed-profile}}

Locations documented by Journey Schedule are, for the time being, only
Operational Locations. The Journey Schedule is not suitable for
representing a speed profile with temporary speed restrictions for
instance.

\hypertarget{data-consistency}{%
\subsubsection{Data consistency}\label{data-consistency}}

It is a user responsibility to check the order and consistency of
journey sections. The ontology will preserve the sequence (using a List
ontology, because OWL2 has no primitive concepts for order), but whether
the sequence \emph{makes sense} must be checked by other means, such as
SWRL rules, or queries (possibly embedded in SHACL), etc.

\hypertarget{list-ontology}{%
\subsubsection{List ontology}\label{list-ontology}}

The List ontology used here is different from the one used in IfcOwl
(the ontology version of Industry Foundation Classes). Later on, one of
the two ontologies may be eliminated in favor of the other.

Original page:
\href{https://github.com/UICrail/CDM-Telematics/wiki/12b-\%E2\%80\%90-Journey-Schedule}{12b-‐-Journey-Schedule.md}

\begin{longtable}[]{@{}
  >{\raggedright\arraybackslash}p{(\columnwidth - 0\tabcolsep) * \real{0.0556}}@{}}
\toprule\noalign{}
\begin{minipage}[b]{\linewidth}\raggedright
\# Journey Schedule properties
\end{minipage} \\
\midrule\noalign{}
\endhead
\bottomrule\noalign{}
\endlastfoot
\# Operational State \\
\#\# Purpose \\
Describe time-dependent attributes (properties) that are relevant to
operations: train run state (``where is my train?''), train state
(``what is the train composition?''), load state (``on which wagon is my
container?''), etc. \\
\#\# Diagram \\
``Operational state'' is the topmost class, on which the time-dependency
hinges. Time can be expressed as an instant or as an interval. \\
Time instant suggests that the state is actually a state change, valid
until the next, or a spot measurement (``is my train at rest?''). \\
Time interval suggests that the state extends over the interval, or
maybe the state change. \\
\includegraphics{https://github.com/UICrail/CDM-Telematics/blob/main/Graphol/diagrams/TAF_revisited_13\%20-\%20Operational\%20State\%20(Situation).png} \\
\#\# Comments \\
\#\#\# Restriction on property dul:hasTimeInterval \\
In our context we use the predefined property dul:hasTimeInterval but
restrict its range to time:Interval (improper time interval in the W3C
Time ontology) when it applies to an Operational State. This is exactly
what the OWL2 restriction does. Note that the range of
dul:hasTimeInterval is dul:TimeInterval, a very broad concept that may
well encompass any time:Interval. We made this explicit in the 99-Varia
diagram. \\
\#\#\# Ambiguity to be lifted \\
The ambiguity state / state change should be removed at a later
stage. \\
\#\#\# Proper vs.~degenerate intervals \\
All time-related classes are borrowed from the W3C Time ontology. \\
Time intervals can be bounded, or open-ended, or degenerate (start
instant = end instant). Again, it would be useful to clarify when to use
an instant and when to use a degenerate interval. One may consider using
class ProperInterval (with end \textgreater{} start) and exclude
degenerate intervals altogether. \\
Original page:
\href{https://github.com/UICrail/CDM-Telematics/wiki/13-\%E2\%80\%90-Operational-State}{13-‐-Operational-State.md} \\
\end{longtable}

\hypertarget{train-run-state}{%
\section{Train run state}\label{train-run-state}}

\hypertarget{purpose-6}{%
\subsection{Purpose}\label{purpose-6}}

\hypertarget{diagram-6}{%
\subsection{Diagram}\label{diagram-6}}

\begin{figure}
\centering
\includegraphics{images/TAF_revisited_13a - Train run state.png}
\caption{Train run state}
\end{figure}

\hypertarget{comments-6}{%
\subsection{Comments}\label{comments-6}}

Original page:
\href{https://github.com/UICrail/CDM-Telematics/wiki/13a-\%E2\%80\%90-Train-run-state}{13a-‐-Train-run-state.md}

\begin{longtable}[]{@{}
  >{\raggedright\arraybackslash}p{(\columnwidth - 0\tabcolsep) * \real{0.0556}}@{}}
\toprule\noalign{}
\begin{minipage}[b]{\linewidth}\raggedright
\# Train state
\end{minipage} \\
\midrule\noalign{}
\endhead
\bottomrule\noalign{}
\endlastfoot
\# Load State \\
\#\# Purpose \\
\#\# Diagram \\
\includegraphics{images/TAF_revisited_13c - Load state.png} \\
\#\# Comments \\
Original page:
\href{https://github.com/UICrail/CDM-Telematics/wiki/13c-\%E2\%80\%90-Load-State}{13c-‐-Load-State.md} \\
\end{longtable}

\hypertarget{message}{%
\section{Message}\label{message}}

\hypertarget{purpose-7}{%
\subsection{Purpose}\label{purpose-7}}

\hypertarget{diagram-7}{%
\subsection{Diagram}\label{diagram-7}}

\begin{figure}
\centering
\includegraphics{images/TAF_revisited_14 - Message.png}
\caption{Message}
\end{figure}

\hypertarget{comments-7}{%
\subsection{Comments}\label{comments-7}}

Original page:
\href{https://github.com/UICrail/CDM-Telematics/wiki/14-\%E2\%80\%90-Message}{14-‐-Message.md}

\begin{longtable}[]{@{}
  >{\raggedright\arraybackslash}p{(\columnwidth - 0\tabcolsep) * \real{0.0556}}@{}}
\toprule\noalign{}
\begin{minipage}[b]{\linewidth}\raggedright
\# Image
\end{minipage} \\
\midrule\noalign{}
\endhead
\bottomrule\noalign{}
\endlastfoot
\# RID codes \\
\#\# Purpose \\
Add the information about dangerous goods and the resulting hazard
classes, as per RID. \\
Information is defined in chapter 3.2 of the RID, Table A, columns 1 and
3a, respectively. \\
\#\# Diagram \\
The information relates to the dangerous substance, but shall be affixed
on the ITU or the wagon, as per chapter 5 of RID. \\
Accordingly, two properties are provided, the domain of which is the
union of Wagon and ITU. This union is a disjoint one, since classes
Wagon and ITU were declared disjoint (no wagon is an ITU, no ITU is a
wagon); it is represented in GRAPHOL by a flattened black hexagon, with
no name (hence a ``blank node''). \\
\includegraphics{images/TAF_revisited_20 - RID.png} \\
\#\# Comments \\
Original page:
\href{https://github.com/UICrail/CDM-Telematics/wiki/20-\%E2\%80\%90-RID-codes}{20-‐-RID-codes.md} \\
\end{longtable}

\hypertarget{time}{%
\section{Time}\label{time}}

\hypertarget{purpose-8}{%
\subsection{Purpose}\label{purpose-8}}

Propose ``time entities'' to be consumed by time-dependent entities
(such as ``train speed'' or ``custom clearance status'').

The time entities are derived from the
\href{https://www.w3.org/TR/owl-time/}{W3C time ontology} and aligned
with the DUL TimeInterval concept.

\hypertarget{diagram-8}{%
\subsection{Diagram}\label{diagram-8}}

An operational time is either an instant or an interval: these are
disjoint classes, as the black flattened hexagon indicates in the
diagram. The distinction between instant and interval rests on
semantics, not on timestamp values: an interval has a beginning and an
end (whether or not the values are known), while an instant has a
beginning (instant) and an end (instant). These may happen to coincide,
i.e.~have equal timestamp values.

The GRAPHOL diagram expresses that each interval is expected to have
exactly one beginning and one end, by means of OWL universal
restrictions (``forAll''), and that the beginning or end are of type
``Operational instant''.

\emph{Note: if the data provide two beginnings for an interval, the
logical consequence is that the beginnings B1 and B2 are the same
individual (:B1 owl:sameAs :B2) and any OWL reasoner will infer that and
inform the user accordingly. Then the user software may want to compare
the respective timestamp values (OWL2 ignores them, so use SPARQL or
code). If timestamps of B1 and B2 fail the test for equality (say,
difference is more than one millisecond? which is context-dependent),
then the user has detected an inconsistency and should consider
resolving it. In a world of imperfect data, wrong data should not break
the database. Ontologies, being robust against such data errors, are a
suitable tool to represent them, but the onus of error detection is
partly on the user: OWL2 has no numeric operators.}

Operational times (instants or intervals) may have a ``date and time of
issue''. This non-mandatory information is of interest in the case of
repeated forecasting or revision exercises. The bulk of exchanged times,
in an operational environment, is composed forecasts or revisions, so
this ``time property of time'' is everything but ludicrous.

\begin{figure}
\centering
\includegraphics{images/TAF_revisited_90 - Time.png}
\caption{Time}
\end{figure}

The bottom part (Temporal role class and individuals) add meaning to
time-dependent situations, i.e.~whether they relate to a planned,
revised, forecast, actual, etc. situation (the DUL term) or state (our
used term).

\hypertarget{comments-8}{%
\subsection{Comments}\label{comments-8}}

\hypertarget{real-time-applications}{%
\subsubsection{Real Time applications}\label{real-time-applications}}

The time representation is conceptually compatible with the
\href{https://www.omg.org/omgmarte/}{UML MARTE profile}, so it can serve
as a basis for a demanding real-time application. The main missing
concept is that of Clock: we currently assume all clocks to be
synchronized, so we ignore them altogether (there is no need for a Clock
class).

We do not distinguish intervals from ``proper intervals'' (length
\textgreater{} 0) in the sense of the W3C Time ontology: as no
assumptions are made regarding the timestamps, if may well happen that
the beginning and the end of an interval are distinct (in actual time),
but have the same timestamp value. Calling it a ``proper interval''
would then be a semantic error.

\hypertarget{against-open-ended-intervals}{%
\subsubsection{Against open-ended
intervals}\label{against-open-ended-intervals}}

Leaving out the beginning or end of an interval \emph{may} be
interpreted as the interval being ``open-ended''. We would like to
sternly warn against such convention, since it breaks the monotonicity
of time reasoning. Example: if a fire extinguisher has a usability
interval but the end is missing from the available data, it shall not be
understood as ``usable forever'', as this conclusion would immediately
be falsified by the provision, at some stage, of the missing
information.

In the context of railway applications, nothing is actually open-ended
in time, and most data may be made available with some delay, so the
``open ended'' interpretation of any interval with missing data would be
a gross mistake.

Original page:
\href{https://github.com/UICrail/CDM-Telematics/wiki/90-\%E2\%80\%90-Time}{90-‐-Time.md}

\begin{longtable}[]{@{}
  >{\raggedright\arraybackslash}p{(\columnwidth - 0\tabcolsep) * \real{0.0556}}@{}}
\toprule\noalign{}
\begin{minipage}[b]{\linewidth}\raggedright
\# Varia
\end{minipage} \\
\midrule\noalign{}
\endhead
\bottomrule\noalign{}
\endlastfoot
\# Dependencies \\
\#\# General and upper ontologies \\
\#\#\# W3C Time ontology \\
\#\#\# SOSA/SSN \\
\#\#\# QUDT \\
\#\#\# DOLCE+DnS Ultralite \\
\#\#\# REGORG, ORG \\
\#\# Semantic RSM ontologies \\
\#\#\# Rolling stock ontologies \\
\#\#\#\# Consist \\
\#\#\#\# Typology \\
\#\# ERA Concept Schemes \\
Original page:
\href{https://github.com/UICrail/CDM-Telematics/wiki/97-\%E2\%80\%90-Dependencies}{97-‐-Dependencies.md} \\
\end{longtable}

\hypertarget{references}{%
\section{References}\label{references}}

\ldots{}

Original page:
\href{https://github.com/UICrail/CDM-Telematics/wiki/99-\%E2\%80\%90-References}{99-‐-References.md}

\end{document}
